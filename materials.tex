% Options for packages loaded elsewhere
\PassOptionsToPackage{unicode}{hyperref}
\PassOptionsToPackage{hyphens}{url}
\PassOptionsToPackage{dvipsnames,svgnames,x11names}{xcolor}
%
\documentclass[
  letterpaper,
  DIV=11,
  numbers=noendperiod]{scrartcl}

\usepackage{amsmath,amssymb}
\usepackage{iftex}
\ifPDFTeX
  \usepackage[T1]{fontenc}
  \usepackage[utf8]{inputenc}
  \usepackage{textcomp} % provide euro and other symbols
\else % if luatex or xetex
  \usepackage{unicode-math}
  \defaultfontfeatures{Scale=MatchLowercase}
  \defaultfontfeatures[\rmfamily]{Ligatures=TeX,Scale=1}
\fi
\usepackage{lmodern}
\ifPDFTeX\else  
    % xetex/luatex font selection
\fi
% Use upquote if available, for straight quotes in verbatim environments
\IfFileExists{upquote.sty}{\usepackage{upquote}}{}
\IfFileExists{microtype.sty}{% use microtype if available
  \usepackage[]{microtype}
  \UseMicrotypeSet[protrusion]{basicmath} % disable protrusion for tt fonts
}{}
\makeatletter
\@ifundefined{KOMAClassName}{% if non-KOMA class
  \IfFileExists{parskip.sty}{%
    \usepackage{parskip}
  }{% else
    \setlength{\parindent}{0pt}
    \setlength{\parskip}{6pt plus 2pt minus 1pt}}
}{% if KOMA class
  \KOMAoptions{parskip=half}}
\makeatother
\usepackage{xcolor}
\setlength{\emergencystretch}{3em} % prevent overfull lines
\setcounter{secnumdepth}{-\maxdimen} % remove section numbering
% Make \paragraph and \subparagraph free-standing
\ifx\paragraph\undefined\else
  \let\oldparagraph\paragraph
  \renewcommand{\paragraph}[1]{\oldparagraph{#1}\mbox{}}
\fi
\ifx\subparagraph\undefined\else
  \let\oldsubparagraph\subparagraph
  \renewcommand{\subparagraph}[1]{\oldsubparagraph{#1}\mbox{}}
\fi


\providecommand{\tightlist}{%
  \setlength{\itemsep}{0pt}\setlength{\parskip}{0pt}}\usepackage{longtable,booktabs,array}
\usepackage{calc} % for calculating minipage widths
% Correct order of tables after \paragraph or \subparagraph
\usepackage{etoolbox}
\makeatletter
\patchcmd\longtable{\par}{\if@noskipsec\mbox{}\fi\par}{}{}
\makeatother
% Allow footnotes in longtable head/foot
\IfFileExists{footnotehyper.sty}{\usepackage{footnotehyper}}{\usepackage{footnote}}
\makesavenoteenv{longtable}
\usepackage{graphicx}
\makeatletter
\def\maxwidth{\ifdim\Gin@nat@width>\linewidth\linewidth\else\Gin@nat@width\fi}
\def\maxheight{\ifdim\Gin@nat@height>\textheight\textheight\else\Gin@nat@height\fi}
\makeatother
% Scale images if necessary, so that they will not overflow the page
% margins by default, and it is still possible to overwrite the defaults
% using explicit options in \includegraphics[width, height, ...]{}
\setkeys{Gin}{width=\maxwidth,height=\maxheight,keepaspectratio}
% Set default figure placement to htbp
\makeatletter
\def\fps@figure{htbp}
\makeatother

\KOMAoption{captions}{tableheading}
\makeatletter
\makeatother
\makeatletter
\makeatother
\makeatletter
\@ifpackageloaded{caption}{}{\usepackage{caption}}
\AtBeginDocument{%
\ifdefined\contentsname
  \renewcommand*\contentsname{Table of contents}
\else
  \newcommand\contentsname{Table of contents}
\fi
\ifdefined\listfigurename
  \renewcommand*\listfigurename{List of Figures}
\else
  \newcommand\listfigurename{List of Figures}
\fi
\ifdefined\listtablename
  \renewcommand*\listtablename{List of Tables}
\else
  \newcommand\listtablename{List of Tables}
\fi
\ifdefined\figurename
  \renewcommand*\figurename{Figure}
\else
  \newcommand\figurename{Figure}
\fi
\ifdefined\tablename
  \renewcommand*\tablename{Table}
\else
  \newcommand\tablename{Table}
\fi
}
\@ifpackageloaded{float}{}{\usepackage{float}}
\floatstyle{ruled}
\@ifundefined{c@chapter}{\newfloat{codelisting}{h}{lop}}{\newfloat{codelisting}{h}{lop}[chapter]}
\floatname{codelisting}{Listing}
\newcommand*\listoflistings{\listof{codelisting}{List of Listings}}
\makeatother
\makeatletter
\@ifpackageloaded{caption}{}{\usepackage{caption}}
\@ifpackageloaded{subcaption}{}{\usepackage{subcaption}}
\makeatother
\makeatletter
\@ifpackageloaded{tcolorbox}{}{\usepackage[skins,breakable]{tcolorbox}}
\makeatother
\makeatletter
\@ifundefined{shadecolor}{\definecolor{shadecolor}{rgb}{.97, .97, .97}}
\makeatother
\makeatletter
\makeatother
\makeatletter
\makeatother
\ifLuaTeX
  \usepackage{selnolig}  % disable illegal ligatures
\fi
\IfFileExists{bookmark.sty}{\usepackage{bookmark}}{\usepackage{hyperref}}
\IfFileExists{xurl.sty}{\usepackage{xurl}}{} % add URL line breaks if available
\urlstyle{same} % disable monospaced font for URLs
\hypersetup{
  colorlinks=true,
  linkcolor={blue},
  filecolor={Maroon},
  citecolor={Blue},
  urlcolor={Blue},
  pdfcreator={LaTeX via pandoc}}

\author{}
\date{}

\begin{document}
\ifdefined\Shaded\renewenvironment{Shaded}{\begin{tcolorbox}[breakable, frame hidden, sharp corners, interior hidden, boxrule=0pt, borderline west={3pt}{0pt}{shadecolor}, enhanced]}{\end{tcolorbox}}\fi

\hypertarget{materials}{%
\subsection{Materials}\label{materials}}

\hypertarget{lecture-notes}{%
\subsubsection{Lecture notes}\label{lecture-notes}}

Lecture notes include Katie's lecture notes and additional resources
from each week, including slides, demos, and further reading.

\begin{itemize}
\tightlist
\item
  \href{notes/week-01-r-basics.qmd}{Week 1: R Basics}
\item
  Week 2: Data visualization
\item
  Week 3: Data wrangling
\item
  Week 4: Sampling distribution
\item
  Week 5: Hypothesis testing
\item
  Week 6: Exam 1 review
\item
  Week 7: Model specification
\item
  Week 8: Model fitting
\item
  Week 9: Model accuracy
\item
  Week 10: Model reliability
\item
  Week 11: Classification
\item
  Week 12: Inference
\item
  Week 13: Exam 2 review
\item
  WeeK 15: Multilevel Models
\end{itemize}

\hypertarget{problem-sets}{%
\subsubsection{Problem sets}\label{problem-sets}}

There are 6 problem sets, due to Gradescope by noon on the following
Mondays. You may request an extension of up to 3 days for any reason.
After solutions are posted, late problem sets can still be submitted for
half credit (50\%). If you submit all 6 problem sets, we will drop your
lowest.

\begin{itemize}
\tightlist
\item
  \href{}{Problem set 1} due Sep 9
\item
  Problem set 2 due Sep 23
\item
  Problem set 3 due Oct 14
\item
  Problem set 4 due Oct 28
\item
  Problem set 5 due Nov 11
\item
  Problem set 6 due Dec 9
\end{itemize}

\hypertarget{exams}{%
\subsubsection{Exams}\label{exams}}

There are 2 midterm exams, taken in class on the following dates. Exams
cannot be rescheduled, except in cases of genuine conflict or emergency
(documentation and a
\href{https://srfs.upenn.edu/student-records/course-action-notices}{Course
Action Notice} are required). However, you can submit any missed exam by
the end of the semester for half credit (50\%). You may also replace
your lowest midterm exam score with the optional final exam.

\begin{itemize}
\tightlist
\item
  Exam 1 in class Tuesday Oct 1
\item
  Exam 2 in class Thursday Nov 21
\item
  Final exam (optional) TBD
\end{itemize}

\hypertarget{lab-exercises}{%
\subsubsection{Lab exercises}\label{lab-exercises}}

Lab exercises are intended for practice and are not graded.

\begin{itemize}
\tightlist
\item
  Lab 1 on Aug 29 or 30
\item
  Lab 2 on Sep 5 or 6
\item
  Lab 3 on Sep 12 or 13
\item
  Lab 4 on Sep 19 or 20
\item
  Lab 5 on Oct 10 or 11
\item
  Lab 6 on Oct 17 or 18
\item
  Lab 7 on Oct 24 or 25
\item
  Lab 8 on Nov 1 or 2
\item
  Lab 9 on Nov 8 or 9
\item
  Lab 10 on Dec 5 or 6
\end{itemize}



\end{document}
